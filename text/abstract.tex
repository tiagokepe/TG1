

The ProInfoData project daily monitors all the computers in Brazil's public schools. The monitoring aims to
provide data to MEC and the society, to monitor the computer's state.

With the growth of the computacinal schools' park and the consequent increase in the volume of data generated,
the original architecture of data storage and query, which is based on a relational model and data warehouse,
appers to be no longer efficient.

To improve system performance we propose a solution that utilizes MapReduce, which is an emerging technology,
but proved to be efficient in various implementations.

The solution we propose in this paper is the transformation of the relational model,
ProIndoData currently employes in the project, to "key-value" model.